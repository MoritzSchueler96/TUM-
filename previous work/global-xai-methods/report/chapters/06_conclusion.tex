%!TEX root = ../main.tex
\section{Discussion, Limitations and Conclusion}
\label{sec:conclusion}
Unfortunately our neural models did not outperform the linear one significantly. 
The linear model is already transparent by design, as the feature weights 
represent directly the influence an input feature has on the result. As we mentioned in
Section~\ref{sec:related_work}, linear models are already used as surrogate for more 
complex models as they are comprehensible for humans. Thus, we did not do any work on 
explainability as our linear model has good performance and could be used as a surrogate
for  the deep neural ones.

We were hoping to find patterns inside a post's text which lead to higher or lower
reaction counts. As the  linear model only considered categorical features and scored
similar performance, this was  not the case. 

The overall higher error of the rtnews dataset can be explained with the dataset's score 
distribution. Rtnews has a higher variance of their posts' score
(Figure~\ref{fig:datasetstats-rtnews}) in comparison to the conspiracy theory group dataset
(Figure~\ref{fig:datasetstats-groups}). 

With only two peak score values as prediction (Figure~\ref{fig:dist-prediction-groups}),
the  linear model almost performed as good as the neural ones. This could be circumvented
by  a more balanced dataset with a uniform distribution of the post reactions count. The
other aspect regarding the dataset is the selection of post sources. We decided to pick
posts  from groups having conspiracy theories as their topic. The relation between score and 
reaction count could behave different in groups or pages with other topics (e.g.
politics).

With the current setup, we could not proof a relation between a post's text and the number
of reactions. There might be no relation at all but we would like to give an idea on what
could still be tried but was not conducted due to time constraints. Facebook  groups
consist of a heterogeneous mix of individuals and most of the posts might not get  much
attention at all (Figure~\ref{fig:datasetstats-groups}). One approach could be to  focus
more on Facebook pages. We only conducted this experiment on a single Facebook page. This
path could lead to more promising results and is yet to be explored.

% \begin{itemize}
%     \item Summarize the insights we obtained
%     \item Problems:
%     \begin{itemize}
%         \item Score variable may not be dependent on the post's text itself
%         \item Unbalanced dataset
%         \item Posts inside groups vary a lot (e.g. off-topic posts vs. viral topic)
%     \end{itemize}

% \end{itemize}